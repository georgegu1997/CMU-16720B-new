\documentclass[11pt]{article}

\usepackage{amsmath}
\usepackage{amssymb}
\usepackage{graphicx}
\usepackage{caption}
\usepackage{subcaption}

\topmargin -.5in
\textheight 9in
\oddsidemargin -.25in
\evensidemargin -.25in
\textwidth 7in

\newcommand{\code}[1]{\texttt{#1}}

\begin{document}

\author{Gu, Qiao}
\title{16-720B Homework 3 Write-up}
\maketitle

\medskip

\subsection*{Q1.1}

\newcommand{\W} {\mathcal{W}}
\newcommand{\I} {\mathcal{I}}
\newcommand{\B} {\mathcal{B}}
\newcommand{\w} {\mathbf{w}}
\newcommand{\x} {\mathbf{x}}
\newcommand{\p} {\mathbf{p}}
\newcommand{\A} {\mathbf{A}}

\begin{itemize}
  \item $\frac{\partial \W(\x; \p)} {\partial \p^T}$ is the graident of the warped coordinates over the warping parameter $\p$, which is:
  \begin{align} \label{warp_gradient_translation}
      \frac{\partial \W(\x; \p)} {\partial \p^T} =
      \frac{\partial \x + \p} {\partial \p^T} =
      \begin{bmatrix}
          1 & 0 \\
          0 & 1
      \end{bmatrix}.
  \end{align}

  \item For the iterative process, replace $\p$ with $\p+\Delta \p$ in Eq. (2) of the handout, and then

  \begin{align} \label{ls}
      \I_{t+1}(\x+\p+\Delta \p) - \I_t(\x)
      &= \I_{t+1} (\x + \p) + \frac{\partial \I_{t+1}(\x+\p)}{\partial(\x+\p)^T} \Delta \p - \I_t(\x) \\
      &= \nabla \I_{t+1}(\x+\p) \Delta \p - (\I_t(\x) - \I_{t+1} (\x+\p)).
  \end{align}

  Therefore the Eq. 2 of the handout in vector form is (Note that each $\nabla \I_{t+1}(\x+\p)$ are of shape $1\times2$.)

  \begin{align} \label{eq:q1.1LS}
      \arg\min_{\Delta \p}
      \left \|
      \begin{bmatrix}
          \nabla \I_{t+1}(\x_1+\p) \\
          \nabla \I_{t+1}(\x_2+\p) \\
          \cdots\\
          \nabla \I_{t+1}(\x_D+\p)
      \end{bmatrix}
      \Delta \p -
      \begin{bmatrix}
          \I_t(\x_1) - \I_{t+1} (\x_1+\p)\\
          \I_t(\x_2) - \I_{t+1} (\x_2+\p)\\
          \cdots\\
          \I_t(\x_D) - \I_{t+1} (\x_D+\p)\\
      \end{bmatrix}
      \right \|
      =
      \arg\min_{\Delta \p}
      \| \A\Delta\p - \mathbf{b} \|
  \end{align}

  The big matrix and the big vector on the L.H.S. of the above equation are the $\A$ and $\mathbf{b}$.

  \item To solve for the least square solution of Eq.~\ref{eq:q1.1LS}, we need to compute $(\A^T\A)^{-1}\A^T\mathbf{b}$. Therefore, we must have $\A^T\A$ to be invertible.
\end{itemize}

\newpage
\subsection*{Q1.3}

Please find the results of Lucas-Kanade tracking results in Figure.~\ref{fig:q1.3}

\begin{figure}[h!]
    \begin{subfigure}{.195\textwidth}
      \centering
      \includegraphics[width=.95\linewidth]{../results/carseqrects_0.png}
      \caption{frame 1}
    \end{subfigure}
    \begin{subfigure}{.195\textwidth}
      \centering
      \includegraphics[width=.95\linewidth]{../results/carseqrects_99.png}
      \caption{frame 100}
    \end{subfigure}
    \begin{subfigure}{.195\textwidth}
      \centering
      \includegraphics[width=.95\linewidth]{../results/carseqrects_199.png}
      \caption{frame 200}
    \end{subfigure}
    \begin{subfigure}{.195\textwidth}
      \centering
      \includegraphics[width=.95\linewidth]{../results/carseqrects_299.png}
      \caption{frame 300}
    \end{subfigure}
    \begin{subfigure}{.195\textwidth}
      \centering
      \includegraphics[width=.95\linewidth]{../results/carseqrects_399.png}
      \caption{frame 400}
    \end{subfigure}\hfill
    \caption{Lucas-Kanade Tracking Results with One Single Template. }
    \label{fig:q1.3}
\end{figure}

\newpage
\subsection*{Q1.4}

Please note that for the implementation for this question, I wrote a function \code{LucasKanadeTrackerWithTemplateCorrection()} in \code{LucasKanade.py}, which handles the routine of Lucas Kanada tracking with template correction.

Please find the results from Lucas-Kanade Tracking with template correction in Figure.~\ref{fig:q1.4}, with the comparison to that without template update. We can clearly see that the tracking with template correction yields a better result for later frames.

\begin{figure}[h!]
    \begin{subfigure}{.195\textwidth}
      \centering
      \includegraphics[width=.95\linewidth]{../results/carseqrects-wrct_0.png}
      \caption{frame 1}
    \end{subfigure}
    \begin{subfigure}{.195\textwidth}
      \centering
      \includegraphics[width=.95\linewidth]{../results/carseqrects-wrct_99.png}
      \caption{frame 100}
    \end{subfigure}
    \begin{subfigure}{.195\textwidth}
      \centering
      \includegraphics[width=.95\linewidth]{../results/carseqrects-wrct_199.png}
      \caption{frame 200}
    \end{subfigure}
    \begin{subfigure}{.195\textwidth}
      \centering
      \includegraphics[width=.95\linewidth]{../results/carseqrects-wrct_299.png}
      \caption{frame 300}
    \end{subfigure}
    \begin{subfigure}{.195\textwidth}
      \centering
      \includegraphics[width=.95\linewidth]{../results/carseqrects-wrct_399.png}
      \caption{frame 400}
    \end{subfigure}\hfill
    \caption{Lucas-Kanade Tracking Results with Template Correction (in yellow boxes). The performance of the baseline tracker in \textbf{Q1.3} is in green boxes. }
    \label{fig:q1.4}
\end{figure}

\newpage
\subsection*{Q2.1}

From the Eq. 6 of the handout, for $w_i$

\begin{align}
    & \B_i^T(\x)(\I_{t+1}(\x) - \I_t(\x)) = \B_i^T(\x)(\sum_{k=1}^K w_k \B_k (\x))\\
    \Rightarrow & \B_i(\x)^T(\I_{t+1}(\x) - \I_t(\x)) = w_i \B_i^T(\x) \B_i(\x) = w_i \\
    \Rightarrow & w_i = \B_i^T(\x) (\I_{t+1}(\x) - \I_t(\x)).
\end{align}

Therefore, for the vector $\w$

\begin{equation}
    \w =
    \begin{bmatrix}
    \B_1^T (\x) \\ \B_2^T (\x) \\ \cdots \\ \B_K^T (\x)
    \end{bmatrix}
    (\I_{t+1}(\x) - \I_t(\x)).
\end{equation}

\newpage
\subsubsection*{Q2.2}

The optimization task can be converted to:

\begin{equation}
    \arg\min_{\Delta \p} \| \mathbf{B}^\bot(\A\Delta \p -\mathbf{b}) \|^2_2
    = \arg\min_{\Delta \p} \| (\mathbf{I}-\mathbf{B}\mathbf{B}^T)\A\Delta\p - (\mathbf{I}-\mathbf{B}\mathbf{B}^T)\mathbf{b}\|^2_2.
\end{equation}

And the above problem can be solved as a least square problem.

\newpage
\subsubsection*{Q2.3}

Please find the results from Lucas-Kanade Tracking with appearance basis in Figure.~\ref{fig:q2.3}, with the comparison to the baseline tracker in \textbf{Q1.3}. We can see that the performance of the two trackers are almost the same, and their bounding boxes basically overlaps with each other. This is probably because as the baseline tracker update its template in every iteration, it can account for some changes of template appearance.

\begin{figure}[h!]
    \begin{subfigure}{.195\textwidth}
      \centering
      \includegraphics[width=.95\linewidth]{../results/sylvseqrects_0.png}
      \caption{frame 1}
    \end{subfigure}
    \begin{subfigure}{.195\textwidth}
      \centering
      \includegraphics[width=.95\linewidth]{../results/sylvseqrects_199.png}
      \caption{frame 200}
    \end{subfigure}
    \begin{subfigure}{.195\textwidth}
      \centering
      \includegraphics[width=.95\linewidth]{../results/sylvseqrects_299.png}
      \caption{frame 300}
    \end{subfigure}
    \begin{subfigure}{.195\textwidth}
      \centering
      \includegraphics[width=.95\linewidth]{../results/sylvseqrects_349.png}
      \caption{frame 350}
    \end{subfigure}
    \begin{subfigure}{.195\textwidth}
      \centering
      \includegraphics[width=.95\linewidth]{../results/sylvseqrects_399.png}
      \caption{frame 400}
    \end{subfigure}\hfill
    \caption{Lucas-Kanade Tracking Results with Appearance Basis (in yellow boxes). The performance of the baseline tracker in \textbf{Q1.3} is in green boxes. }
    \label{fig:q2.3}
\end{figure}

\newpage
\subsection*{Q3.1}

Previously, for only translation, the gradient of warping function $\W$ w.r.t the parameters $\p$, is in Eq. \ref{warp_gradient_translation}. Now, since

\begin{align}
    \W(\x; \p) &=
    \begin{bmatrix}
        (1+p_1)x+p_2 y+p_3 \\
        p_4 x+(1+p_5) y +p_6
    \end{bmatrix}
    \\ \Rightarrow
    \frac{\partial \W(\x; \p)} {\partial \p^T} &=
    \begin{bmatrix}
        x&y&1&0&0&0 \\
        0&0&0&x&y&1
    \end{bmatrix}.
\end{align}

Now go back to the Eq. 4 in the handout and Eq. \ref{ls}:
\newcommand {\Wxdp} {\W(\x;\p+\Delta \p)}
\newcommand {\Wxp} {\W(\x;\p)}

\begin{align} \label{general_warping_expansion}
    \I_{t+1}(\Wxdp)
    & \approx
    \I_{t+1}(\x')+
    \frac{\partial \I_{t+1}(\x')}{\partial(\x')^T}
    \frac{\partial \W(\x; \p)} {\partial \p^T} \Delta \p
    \\ \Rightarrow
    \I_{t+1}(\Wxdp) - \I_t(\x)
    & \approx
    \I_{t+1} (\x') +
    \frac {\partial \I_{t+1}(\x')} {\partial(\x')^T}
    \frac{\partial \W(\x; \p)} {\partial \p^T}
    \Delta \p - \I_t(\x) \\
    &= \nabla \I_{t+1}(\x')
    \frac{\partial \W(\x; \p)} {\partial \p^T}
    \Delta \p - (\I_t(\x) - \I_{t+1} (\x')),
\end{align}

where $\x'=\Wxp$. Therefore, the optimzation object in Eq. \ref{eq:q1.1LS} becomes:

\begin{equation} \label{general_optimization_object}
    \arg\min_{\Delta \p}
    \left \|
    \begin{bmatrix}
        \nabla \I_{t+1}(\x_1') \frac{\partial\W(\x_1;\p)}{\partial \p^T} \\
        \nabla \I_{t+1}(\x_2') \frac{\partial\W(\x_2;\p)}{\partial \p^T} \\
        \cdots\\
        \nabla \I_{t+1}(\x_D') \frac{\partial\W(\x_D;\p)}{\partial \p^T}
    \end{bmatrix}
    \Delta \p -
    \begin{bmatrix}
        \I_t(\x_1) - \I_{t+1} (\x_1')\\
        \I_t(\x_2) - \I_{t+1} (\x_2')\\
        \cdots\\
        \I_t(\x_D) - \I_{t+1} (\x_D')\\
    \end{bmatrix}
    \right \|
    =
    \arg\min_{\Delta \p}
    \| \A_{affine}\Delta\p - \mathbf{b} \|
\end{equation}


\end{document}
