\documentclass[11pt]{article}

\usepackage{amsmath}
\usepackage{amssymb}
\usepackage{graphicx}
\usepackage{caption}
\usepackage{subcaption}

\topmargin -.5in
\textheight 9in
\oddsidemargin -.25in
\evensidemargin -.25in
\textwidth 7in

\newcommand{\code}[1]{\texttt{#1}}

\begin{document}

\author{Gu, Qiao}
\title{16-720B Homework 4 Write-up}
\maketitle

\medskip

\subsection*{Q1.1}

\newcommand{\intrinsic}{\mathbf{K}}
\newcommand{\fundamental}{\mathbf{F}}
\newcommand{\essential}{\mathbf{E}}
\newcommand{\homox}{\tilde{\mathbf{x}}}

Consider the point $\mathbf{w}$ where the principle axes of the two cameras intersect, and we can see that $\homox_1=[0,0,1]^T$ and $\homox_2=[0,0,1]^T$ corresponding one point in 3D. Therefore

\begin{align}
    \homox_2^T \essential \homox_1 =
    \begin{bmatrix}
        0 & 0 & 1
    \end{bmatrix}
    \begin{bmatrix}
        \essential_{11} & \essential_{12} & \essential_{13} \\
        \essential_{21} & \essential_{22} & \essential_{23} \\
        \essential_{31} & \essential_{32} & \essential_{33}
    \end{bmatrix}
    \begin{bmatrix}
        0 \\ 0 \\ 1
    \end{bmatrix}
    = \essential_{33} = 0
\end{align}

Since two cameras are normalized, the intrinsic matrices for them are identity:  $\intrinsic_1=\intrinsic_2=\mathbf{I}$.
Then $\essential = \intrinsic_1^T \fundamental \intrinsic_2 = \essential$.
Therefore, $\essential_{33} = \fundamental_{33}=0$.

% Suppose the intrinsic matrix for two cameras are
%
% \begin{align}
%     \intrinsic_1 =
%     \begin{bmatrix}
%         f_{1x} & \gamma_1 & 0 \\
%         0 & f_{1y} & 0 \\
%         0 & 0 & 1
%     \end{bmatrix}
%     \quad
%     \intrinsic_2 =
%     \begin{bmatrix}
%         f_{2x} & \gamma_2 & 0 \\
%         0 & f_{2y} & 0 \\
%         0 & 0 & 1
%     \end{bmatrix}
% \end{align}

% \begin{align}
%     \essential = \intrinsic_1^T \fundamental \intrinsic_2 &=
%     \begin{bmatrix}
%         f_{1x} & 0 & 0 \\
%         \gamma_1 & f_{1y} & 0 \\
%         0 & 0 & 1
%     \end{bmatrix}
%     \begin{bmatrix}
%         \fundamental_{11} & \fundamental_{12} & \fundamental_{13} \\
%         \fundamental_{21} & \fundamental_{22} & \fundamental_{23} \\
%         \fundamental_{31} & \fundamental_{32} & \fundamental_{33}
%     \end{bmatrix}
%     \begin{bmatrix}
%         f_{2x} & \gamma_2 & 0 \\
%         0 & f_{2y} & 0 \\
%         0 & 0 & 1
%     \end{bmatrix}
%     \\
%     &= \begin{bmatrix}
%         f_{1x} & 0 & 0 \\
%         \gamma_1 & f_{1y} & 0 \\
%         0 & 0 & 1
%     \end{bmatrix}
%     \begin{bmatrix}
%         \dots & \dots & \fundamental_{13} \\
%         \dots & \dots & \fundamental_{23} \\
%         \dots & \dots & \fundamental_{33} \\
%     \end{bmatrix}
%     =
%     \begin{bmatrix}
%         \dots & \dots & \dots \\
%         \dots & \dots & \dots \\
%         \dots & \dots & \fundamental_{33} \\
%     \end{bmatrix}
% \end{align}

\newpage

\subsection*{Q1.2}

\newcommand {\rotation} {\mathbf{R}}
\newcommand {\translation} {\mathbf{t}}
\newcommand {\bl} {\mathbf{l}}

Suppose the cameras are normalized in the sense that their intrinsic matrices are both identity:  $\intrinsic_1=\intrinsic_2=\mathbf{I}$.

Now that the translation and rotation from camera 1 to camera 2 are

\begin{align}
    \rotation =
    \begin{bmatrix}
        1 & 0 & 0 \\
        0 & 1 & 0 \\
        0 & 0 & 1
    \end{bmatrix}, \quad
    \translation =
    \begin{bmatrix}
        t_x \\ 0 \\ 0
    \end{bmatrix}
\end{align}

And thus the essential matrix are

\begin{align}
  \essential=\translation_\times \rotation=
  \begin{bmatrix}
    0 & 0 & 0 \\
    0 & 0 & -t_x \\
    0 & t_x & 0
  \end{bmatrix}
\end{align}

Therefore for an epipolar line in camera 1 $\bl_1^T \homox_1=0$ and $\homox_2^T \essential \homox_1 =0$, where $\homox_2$ is a fixed point on the image plane of camera 2 resulting from the ray corresponding to the epipolar line, then we can see that

\begin{align}
  \bl_1^T = \homox_2^T \essential =
  \begin{bmatrix}
    x_2 & y_2 & 1
  \end{bmatrix}
  \begin{bmatrix}
    0 & 0 & 0 \\
    0 & 0 & -t_x \\
    0 & t_x & 0
  \end{bmatrix}
  =
  \begin{bmatrix}
    0 & t_x & -t_x y_2
  \end{bmatrix}
\end{align}

Similarly we can see that any epipolar line in camera 1 has $\bl_2^T = [0 \; -t_x \; t_xy_1]$. Since the first elements in both $\bl_1$ and $\bl_2$ are zero, the epipolar lines are parallel to $x$ axis.

% According to the dfinition of essential matrix
%
% \begin{align}
%     \homox_2^T \translation \times \rotation \homox_1 &=
%     \homox_2^T \translation_\times \rotation \homox_1 =
%     \homox_2^T
%     \begin{bmatrix}
%         0 & 0 & 0 \\
%         0 & 0 & -t_x \\
%         0 & t_x & 0 \\
%     \end{bmatrix}
%     \begin{bmatrix}
%         1 & 0 & 0 \\
%         0 & 1 & 0 \\
%         0 & 0 & 1
%     \end{bmatrix}
%     \homox_1
%     \\ &=
%     \homox_2^T
%     \begin{bmatrix}
%         0 & 0 & 0 \\
%         0 & 0 & -t_x \\
%         0 & t_x & 0 \\
%     \end{bmatrix}
%     \homox_1 =
%     \begin{bmatrix}
%         x_2 & y_2 & 1
%     \end{bmatrix}
%     \begin{bmatrix}
%         0 \\ -t_x \\ t_x y_1
%     \end{bmatrix}
%     \\ &=
%     -t_x y_2 + t_x y_1 = 0
%     \\ & \Rightarrow
%     y_1 = y_2
% \end{align}
%
% For a certain epipolar line in camera 2, it has a fixed epipole with fixed $y_1$ in camera 1, and thus every 3D point on the corresponding incident ray to camera 1 has a projection with a fixed $y_2=y_1$ in camera 2, which means the epipolar line has a fixed $y$ coordinate, and thus parallel to x-axis.
%
% The above deduction also holds if camera 1 and 2 are exchanged.

\newpage

\subsection*{Q1.3}

\newcommand{\pointw}{\mathbf{w}}

Assume $(\rotation_i, \translation_i)$ and $(\rotation_i, \translation_i)$ are the rotation and translation from the world coordinate frame to the camera coordinate frame at time $i$ and time $j$. And suppose $\rotation_{rel}$ and $\translation_{rel}$ are the rotation and translation from camera at time $i$ to the camera at time $j$. Then for a point $\pointw$ in the 3D world

\begin{align}
    & \lambda_i \homox_i = \rotation_i \pointw + \translation_i, \quad
    \lambda_j \homox_j = \rotation_j \pointw + \translation_j
    \nonumber \\ \Rightarrow &
    \pointw = \rotation_i ^T (\lambda_i \homox_i - \translation_i)
    \nonumber \\ \Rightarrow &
    \lambda_j \homox_j = \rotation_j \rotation_i ^T (\lambda_i \homox_i - \translation_i) + \translation_j
    \nonumber \\ \Rightarrow &
    \lambda_j \homox_j = \rotation_j \rotation_i ^T \lambda_i \homox_i - \rotation_j \rotation_i ^T \translation_i + \translation_j
    \nonumber \\ \Rightarrow &
    \lambda_j \homox_j = \lambda_i \rotation_{rel} \homox_i + \translation_{rel}
\end{align}

Therefore

\begin{align}
    \rotation_{rel} = \rotation_j \rotation_i ^T, \quad
    \translation_{rel} = \translation_j - \rotation_j \rotation_i ^T \translation_i
\end{align}

Then the essential and fundamental matrix can be derived as

\begin{align}
  \essential &= (\translation_{rel})_\times \rotation_{rel} \\
  \fundamental &= (\intrinsic^{-1})^T \fundamental \intrinsic^{-1} = (\intrinsic^{-1})^T (\translation_{rel})_\times \rotation_{rel} \intrinsic^{-1}
\end{align}

\newpage

\subsection*{Q1.4}

\newcommand {\bv} {\mathbf{v}}
\newcommand {\bH} {\mathbf{H}}
\newcommand {\bI} {\mathbf{I}}
\newcommand {\bw} {\mathbf{w}}

Suppose the mirror is orthogonal to a unit vector $\bv$ pointing in to the mirror, then we can have the Householder transformation matrix $\bH=\bI - 2\bv\bv^T$. Suppose the real world coordinate has its origin at a point on the mirror, then for any point $\bw$ in the world coordinate, the mirror produce its reflection $\bw_2=\bH\bw_1=\bw_1 - 2\bv\bv^T\bw_1=\bw_1 + 2\alpha\bv$, where $\alpha=-\bv^T\bw_1$ is the dixed distance from $\bw_2$ to the mirror.

This two points in 3D produce two point on the image plane as follows

\begin{align}
  \lambda_1\homox_1 &= \bw_1 \\
  \lambda_2\homox_2 &= \bw_2 = \bw_1 + 2\alpha\bv
\end{align}

This is equivalent to a two-camera system where $\rotation=\bI$ and $\translation=2\alpha\bv$. Therefore

\begin{align}
  \essential = \translation_\times \rotation = 2\alpha\bv_\times \bI=2\alpha\bv_\times
\end{align}

is skew-symmetric. Since there are only one camera with only one intrinsic $\intrinsic$, for fundamental matrix

\begin{align}
  \fundamental = (\intrinsic^{-1})^T \essential \essential^{-1}
  \fundamental^T = (\intrinsic^{-1})^T \essential^T \essential^{-1} = -(\intrinsic^{-1})^T \essential \essential^{-1} = -\fundamental
\end{align}

Therefore the fundamental matrix $\fundamental$ of this equivalent two-camera system is symmetric.

\newpage

\subsection*{Q2.1}

The fundamental matrix $\fundamental$ given by the 8-point algorithm is

\begin{verbatim}
[[ 9.80213861e-10 -1.32271663e-07  1.12586847e-03]
 [-5.72416248e-08  2.97011941e-09 -1.17899320e-05]
 [-1.08270296e-03  3.05098538e-05 -4.46974798e-03]].
\end{verbatim}

And visualization result is shown in Figure.~\ref{fig:q2.1}

\begin{figure}[h!]
    \centering
    \includegraphics[width=.8\linewidth]{../results/q2_1.png}
    \caption{The visualization results showing the fundamental matrix given by 8-point algorithm. }
    \label{fig:q2.1}
\end{figure}

\newpage

\subsection*{Q2.2}

To expedite the process of searching for the best $F$, I iteratively choose 7 points randomly and keep $F$ with the minimum 2-norm difference from $F$ given by 8-point algorithm. And the best reasonable $F$ is given when I use the points at indices [ 10   3  92 108  41  30  99], which is

\begin{verbatim}
[[-1.29290341e-08  1.81485505e-07  8.27390734e-04]
 [-3.52772006e-07  1.05936161e-09  4.14750371e-05]
 [-7.87641850e-04 -1.90120686e-05 -4.69857949e-03]]
\end{verbatim}

And the visualization result is shown in Figure.~\ref{fig:q2.2}

\begin{figure}[h!]
    \centering
    \includegraphics[width=.8\linewidth]{../results/q2_2.png}
    \caption{The visualization results showing the fundamental matrix given by 7-point algorithm. }
    \label{fig:q2.2}
\end{figure}

\newpage

\subsection*{Q3.1}

By applying the equation $\essential=\intrinsic_2^T \fundamental \intrinsic_1$, we can get the $\essential=$

\begin{verbatim}
[[ 2.26587820e-03 -3.06867395e-01  1.66257398e+00]
 [-1.32799331e-01  6.91553934e-03 -4.32775554e-02]
 [-1.66717617e+00 -1.33444257e-02 -6.72047195e-04]]
\end{verbatim}

\newpage

\subsubsection*{Q3.2}

\newcommand {\camera} {\mathbf{C}}
\newcommand {\homow} {\tilde{\mathbf{w}}}
\newcommand {\ct} {\camera^T}
\newcommand {\bA} {\mathbf{A}}

Suppose $\homow$ is the homogenuous coordinate of the 3D point $\bw$, and it projects $\homox_{i1}$ and $\homox_{i2}$ on camera 1 and camera 2 respectively, which means

\begin{align}
    \camera_1 \homow_i = \lambda_1\homox_{i1}, \quad \camera_2 \homow_i = \lambda_2\homox_{i2}
\end{align}

Now only consider $\camera_1 \homow = \lambda_1\homox_{i1}$. Suppose $\homox_{i1} = [x_{i1}, y_{i1}, 1]^T$ and $\ct_{11}, \ct_{12}, \ct_{13}$ are the first, the second, and the third row of the camera matrix $\camera_1$. We get

\begin{align}
    \begin{cases}
        \lambda_1 x_{i1} &= \ct_{11} \homow_i \\
        \lambda_1 y_{i1} &= \ct_{12} \homow_i \\
        \lambda_1 &= \ct_{13} \homow_i
    \end{cases}
    \Rightarrow
    \begin{cases}
        x_{i1} \ct_{13} \homow_i &= \ct_{11} \homow_i \\
        y_{i1} \ct_{13} \homow_i &= \ct_{12} \homow_i
    \end{cases}
    \Rightarrow
    \begin{bmatrix}
        x_{i1} \ct_{13} - \ct_{11}\\
        y_{i1} \ct_{13} - \ct_{12}
    \end{bmatrix}
    \homow_i = \mathbf{0}
\end{align}

We can get Similar constraints from the projection on camera 2, and by concatenate the constraints together we can get $\bA_i\bw_i=0$ as follows:

\begin{align}
    \bA_i\bw_i =
    \begin{bmatrix}
        x_{i1} \ct_{13} - \ct_{11}\\
        y_{i1} \ct_{13} - \ct_{12}\\
        x_{i2} \ct_{23} - \ct_{21}\\
        y_{i2} \ct_{23} - \ct_{22}
    \end{bmatrix}
    \homow_i = \mathbf{0}
    \textrm{ and }
    \bA_i =
    \begin{bmatrix}
        x_{i1} \ct_{13} - \ct_{11}\\
        y_{i1} \ct_{13} - \ct_{12}\\
        x_{i2} \ct_{23} - \ct_{21}\\
        y_{i2} \ct_{23} - \ct_{22}
    \end{bmatrix}
\end{align}

Then $\homow_i$ is in the null space of $\bA_i$, which we can get by solving SVD decomposition of $\bA_i$ and getting the last column of $V$.

% Multiply each side with $\homox_{i1}$ and we get
%
% \begin{align}
%     \homox_{i1} \times \camera_1 \homow_i &= \mathbf{0}
%     \\ \Rightarrow  (\homox_{i1})_\times \camera_1 \homow_i &= \mathbf{0}
%     \\ \Rightarrow
%     \begin{bmatrix}
%         0 & -1 & y_{i1} \\
%         1 & 0 & -x_{i1} \\
%         -y_{i1} & x_{i1} & 0 \\
%     \end{bmatrix}
%     \begin{bmatrix}
%         \ct_{11} \\ \ct_{12} \\ \ct_{13}
%     \end{bmatrix}
%     \homow_i & = \mathbf{0}
%     \\ \Rightarrow
%     \begin{bmatrix}
%         -\ct_{12} + y_{i1}\ct_{13} \\
%         \ct_{11} - x_{i1}\ct_{13} \\
%         -y_{i1}\ct_{11} + x_{i1}\ct_{12}
%     \end{bmatrix}
%      \homow_i & = \mathbf{0}
% \end{align}
%
% Where $\homox_{i1} = [x_{i1}, y_{i1}, 1]^T$ and $\ct_{11}, \ct_{12}, \ct_{13}$ are the first, the second, and the third row of the camera matrix $\camera_1$.

\newpage

\subsection*{Q4.1}

SOme detected correspondences are shown in Figure.~\ref{fig:q4.1}

\begin{figure}[h!]
    \centering
    \includegraphics[width=.8\linewidth]{../results/q4_1.png}
    \caption{Visualization of some detected correspondences. }
    \label{fig:q4.1}
\end{figure}

\newpage

\subsection*{Q4.2}

The recovered point could can be viewed in Figure.~\ref{fig:q4.1}.

\begin{figure}[h!]
    \begin{subfigure}{.49\textwidth}
      \centering
      \includegraphics[width=.95\linewidth]{../results/q4_2_1.png}
      \caption{view 1}
    \end{subfigure}
    \begin{subfigure}{.49\textwidth}
      \centering
      \includegraphics[width=.95\linewidth]{../results/q4_2_2.png}
      \caption{view 2}
    \end{subfigure}\hfill
    \begin{subfigure}{.49\textwidth}
      \centering
      \includegraphics[width=.95\linewidth]{../results/q4_2_3.png}
      \caption{view 3}
    \end{subfigure}
    \begin{subfigure}{.49\textwidth}
      \centering
      \includegraphics[width=.95\linewidth]{../results/q4_2_4.png}
      \caption{view 4}
    \end{subfigure}\hfill
    \caption{Different views of the point cloud recovered from \code{templeCoords}. }
    \label{fig:q4.1}
\end{figure}

\newpage

\subsection*{Q5.1}

As suggested by the lecture notes: When the fundamental matrix is correct, the epipolar line induced by a point in the first image should pass through the matching point in the second image and vice-versa.
Therefore, we used the distance of points to corresponding epipolar lines as the criterion in deciding inliers during RANSAC process. More specifically

\begin{align}
  Err(\homox_2, \fundamental, \homox_1) = dist^2(\homox_2, \fundamental \homox_1) + dist^2(\homox_1, \fundamental^T \homox_2),  \\
  \textrm{where } dist(\homox_2, \fundamental \homox_1) = \frac{\homox_2^T \fundamental \homox_1}{\sqrt{l_1^2 + l_2^2}}, \fundamental \homox_1=[l_1, l_2, l_3]^T
\end{align}

And a correspondence is considered an inlier if its error is below a threshold $\epsilon$. To decide $\epsilon$, I tried different values and found that when $\epsilon=2$, the RANSAC yield around $140*0.75=105$ inliers, and thus decide the threshold to be 2.

\newpage

\subsection*{Q5.3}

The images showing the 3D points before and after bundle adjustment are shown in Figure.~\ref{fig:q5.3}. And in my experiement, the reprojection error before the bundle adjustment is $386.57$ and the error after the bundle adjustment is $8.93$.

\begin{figure}[h!]
    \begin{subfigure}{\textwidth}
      \centering
      \includegraphics[width=.8\linewidth]{../results/q5_3_1.png}
      \caption{view 1}
    \end{subfigure}
    \begin{subfigure}{\textwidth}
      \centering
      \includegraphics[width=.8\linewidth]{../results/q5_3_2.png}
      \caption{view 2}
    \end{subfigure}
    \begin{subfigure}{\textwidth}
      \centering
      \includegraphics[width=.8\linewidth]{../results/q5_3_3.png}
      \caption{view 3}
    \end{subfigure}
    \caption{Different views of the point cloud before and after the bundle adjustment. }
    \label{fig:q5.3}
\end{figure}


\end{document}
